\documentclass{article}

\title{CLS: an approach for a new statistical system}
\author{AJ Rossini}
\date{\today}

\begin{document}

\maketitle

\section{Introduction}
\label{sec:intro}

Statisticians who use a computer for data analysis invariably take one
of two approaches (considered in the extremes here for illustration): 
\begin{enumerate}
\item the \emph{FORTRAN} approach of coding numerical and algorithmic
  information into the computer program code used for the data
  analysis, or 
\item the \emph{GUI} approach, via Microsoft Excel, SPSS, Minitab, and
  similar approaches, where tasks are facilitated, sometimes with
  accompanying workflow support.
\end{enumerate}
Both approaches have co-existed since the early 80s, with the FORTRAN
approach dating back to the dawn of the computing era.

\section{Components of a procedure}
\label{sec:components}

define a statistical procedure as a decision-making approach which
entails the intertwining of formal and informal structure.   

Components:
\begin{enumerate}
\item \label{statproc-decision} Decision to make
\item \label{statproc-assessment} Assessment approach to use
  (some are inherently different, others just look different)
\item \label{statproc-normalization} Normalization of the problem for
  assessment/comparison with other reference behaviours
\item \label{conclusion} Type of conclusion desired, and instance of
  that conclusion (when data is present)
\end{enumerate}

This forms an \textit{abstract class} of a procedure, which can be
represented by a real class, which can then be instantiated through
the application of data.



\end{document}
