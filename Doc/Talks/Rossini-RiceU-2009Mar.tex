\documentclass{beamer}

\mode<presentation>
{
  \usetheme{classic}
  \setbeamercovered{transparent}
}

\usepackage[english]{babel}
\usepackage[latin1]{inputenc}
\usepackage{times}
\usepackage[T1]{fontenc}

\title[CLS]{Common Lisp Statistics}
\subtitle{Using History to design better data analysis environments}
\author[Rossini]{Anthony~(Tony)~Rossini}

\institute[Novartis and University of Washington] % (optional, but mostly needed)
{
  Group Head, Modeling and Simulation\\
  Novartis Pharma AG, Switzerland
  \and
  Affiliate Assoc Prof, Biomedical and Health Informatics\\
  University of Washington, USA}

\date[Rice 09]{Rice, Mar 2009}
\subject{Statistical Computing Environments}

\begin{document}

\begin{frame}
  \titlepage
\end{frame}

\begin{frame}{Outline}
  \tableofcontents
\end{frame}

% Structuring a talk is a difficult task and the following structure
% may not be suitable. Here are some rules that apply for this
% solution: 

% - Exactly two or three sections (other than the summary).
% - At *most* three subsections per section.
% - Talk about 30s to 2min per frame. So there should be between about
%   15 and 30 frames, all told.

% - A conference audience is likely to know very little of what you
%   are going to talk about. So *simplify*!
% - In a 20min talk, getting the main ideas across is hard
%   enough. Leave out details, even if it means being less precise than
%   you think necessary.
% - If you omit details that are vital to the proof/implementation,
%   just say so once. Everybody will be happy with that.

\section{Preliminaries}

\subsection{Context}

\begin{frame}{Goals for this Talk}{(define, strategic approach,
    justify)}

  \begin{itemize}
  \item To describe the concept of \alert{computable and executable
      statistics}, placing it in a historical context.

  \item To demonstrate that \alert{a research program}
    implemented through  simple steps can increase the efficiency  of
    statistical computing approaches by  clearly describing both:
    \begin{itemize}
    \item numerical characteristics of procedures,
    \item statistical concepts driving them.
    \end{itemize}

  \item To justify that the \alert{approach is worthwhile} and
    represents a staged effort towards \alert{increased use of best
      practices}.
  \end{itemize}
  (unfortunately, the last is still incomplete)
\end{frame}


\begin{frame}{Historical Computing Languages}
  \begin{itemize}
  \item FORTRAN : FORmula TRANslator.  Original numerical computing
    language, designed for clean implementation of numerical
    algorithms
  \item LISP : LISt Processor.  Associated with symbolic
    manipulation, AI, and knowledge approaches
  \end{itemize}

  They represent the 2 generalized needs of statistical computing,
  which could be summarized as
  \begin{itemize}
  \item algorithms/numerics,
  \item elicitation, communication, and generation of knowledge (``data
    analysis'')
  \end{itemize}
\end{frame}

\begin{frame}{Statistical Computing Environments}

  Past: 
  \begin{itemize}
  \item SPSS / BMDP / SAS
  \item S ( S, S-PLUS, R)
  \item LispStat ( XLispStat,  ViSta, ARC , CommonLispStat ) ; QUAIL
  \item XGobi (Orca / GGobi / Statistical Reality Engine)
  \item MiniTab
  \item Stata
  \item DataDesk
  \item Augsburg Impressionist series (MANET, 
  \item Excel
  \end{itemize}
  many others...

\end{frame}

\begin{frame}{How many are left?}

  \begin{itemize}
  \item R 
  \item SAS
  \item SPSS
  \item Stata
  \item Minitab
  \item very few others...    
  \end{itemize}
  ``R is the Microsoft of the statistical computing world'' -- anonymous.
\end{frame}

\begin{frame}{Selection Pressure}
  \begin{itemize}
  \item the R user population is growing rapidly, fueled by critical
    mass, quality, and value
  \item R is a great system for applied data analysis
  \item R is not such a great system for research into statistical
    computing (backwards compatibility, inertia due to user population)
  \end{itemize}
  There is a need for alternative experiments for developing new
  approaches/ideas/concepts. 
\end{frame}

\begin{frame}{Philosophically, why Common Lisp?}
  Philosophically:
  \begin{itemize}
  \item Lisp can cleanly present computational intentions, both
    symbolically and numerically.
  \item Semantics and context are important: well supported by Lisp
    paradigms.
  \item Lisp's parentheses describe singular, multi-scale,
    \alert{complete thoughts}.
  \end{itemize}

\end{frame}

\begin{frame}{Technically, why Common Lisp?}
  \begin{itemize}
  \item interactive COMPILED language (``R with a compiler'')
  \item CLOS is R's S4 object system ``done right''.
  \item clean semantics: modality, typing, can be expressed the way
    one wants it.
  \item programs are data, data are programs, leading to
  \item Most modern computing tools available (XML, WWW technologies)
  \item ``executable XML''
  \end{itemize}
  Common Lisp is very close in usage to how people currently use R
  (mostly interactive, some batch, and a wish for compilation efficiency).
\end{frame}

\subsection{Background}

\begin{frame}
  \frametitle{Desire: Semantics and Statistics}
  \begin{itemize}
  \item The semantic web (content which is self-descriptive) is an
    interesting and potentially useful idea.
    
  \item 
    Biological informatics support (GO, Entrez) has allowed for
    precise definitions of concepts in biology.

  \item It is a shame that a field like statistics, requiring such
    precision, has less than an imprecise and temporally instable
    field such as biology\ldots
  \end{itemize}

  How can we express statistical work (research, applied work) which
  is both human and computer readable (perhaps subject to
  transformations first)?
\end{frame}


% \subsection{Context}

% \begin{frame}{Context}{(where I'm coming from, my ``priors'')}
%   \begin{itemize}
%   \item Pharmaceutical Industry
%   \item Modeling and Simulation uses mathematical models/constructs to
%     record beliefs (biology, pharmacology, clinical science) for
%     explication, clinical team alignment, decision support, and
%     quality.
%   \item My work at Novartis is at the intersection of biomedical
%     informatics, statistics, and mathematical modeling.
%   \item As manager: I need a mix of applications and novel research development to
%     solve our challenges better, faster, more efficiently.
%   \item Data analysis is a specialized approach to computer
%     programming, \alert{different} than applications programming or
%     systems programming.
%   \end{itemize}
% \end{frame}

\section{Computable and Executable Statistics}

\begin{frame}{Can we compute with them?}
  3 Examples:
  \begin{itemize}
  \item Research
  \item Consulting
  \item Reimplementation
  \end{itemize}
  Consider whether one can ``compute'' with the information given?
\end{frame}

\begin{frame}[fragile]{Example 1: Theory\ldots}
  \label{example1}
  Let $f(x;\theta)$ describe the likelihood of XX under the following
  assumptions.  
  \begin{enumerate}
  \item assumption-1
  \item assumption-2
  \end{enumerate}
  Then if we use the following algorithm:
  \begin{enumerate}
  \item step-1
  \item step-2
  \end{enumerate}
  then $\hat{\theta}$ should be $N(0,\hat\sigma^2)$ with the following
  characteristics\ldots
\end{frame}

\begin{frame}
  \frametitle{Can we compute, using this description?}
  Given the information at hand:
  \begin{itemize}
  \item we ought to have a framework for initial coding for the
    actual simulations (test-first!)
  \item the implementation is somewhat clear
  \item We should ask: what theorems have similar assumptions?
  \item We should ask: what theorems have similar conclusions but
    different assumptions?
  \end{itemize}
\end{frame}

\begin{frame}[fragile]{Realizing Theory}
  \small{
\begin{verbatim}  
(define-theorem my-proposed-theorem
   (:theorem-type '(distribution-properties
                    frequentist
                    likelihood))
   (:assumes '(assumption-1 assumption-2))
   (:likelihood-form
      (defun likelihood (data theta gamma)
        (exponential-family theta gamma)))
   (:compute-by
      '(progn
         (compute-starting-values thetahat gammahat)
         (until (convergence)
           (setf convergence
                 (or (step-1 thetahat)
                     (step-2 gammahat))))))
   (:claim (assert 
             (and (equal-distribution thetahat 'normal)
                  (equal-distribution gammahat 'normal)))))
\end{verbatim}
  }
\end{frame}

\begin{frame}[fragile]{It would be nice to have}
\begin{verbatim}
   (theorem-veracity 'my-proposed-theorem)
\end{verbatim}
\end{frame}

\begin{frame}[fragile]{and why not...?}
\begin{verbatim}
   (when (theorem-veracity
              'my-proposed-theorem)
      (write-paper 'my-proposed-theorem
                   :style :JASA
                   :output-format
                         '(LaTeX MSWord)))
\end{verbatim}
\end{frame}

\begin{frame}{Comments}
  \begin{itemize}
  \item The general problem is very difficult
  \item Some progress has been made in small areas of basic
    statistics: currently working on linear regression (LS-based,
    Normal-bayesian) and the T-test.
  \item Areas targetted for medium-term future: resampling methods and
    similar algorithms.
  \end{itemize}

\end{frame}

\begin{frame}
  \frametitle{Example 2: Practice\ldots} 
  \label{example2}
  The dataset comes from a series of clinical trials.  We model the
  primary endpoint, ``relief'', as a binary random variable.  There is
  a random trial effect on relief as well as severity due to
  differences in recruitment and inclusion/exclusion criteria.
\end{frame}

\begin{frame}
  \frametitle{Can we compute, using this description?}
  \begin{itemize}
  \item With a real such description, it is clear what some of the
    potential models might be for this dataset
  \item It should be clear how to start thinking of a data dictionary
    for this problem.
  \end{itemize}
\end{frame}

\begin{frame}[fragile]{Can we compute?}
\begin{verbatim}
  (dataset-metadata paper-1
    :context 'clinical-trials
    :variables '((relief :model-type dependent
                         :distribution binary)
                 (trial  :model-type independent
                         :distribution categorical)
                 (disease-severity))
    :metadata '(inclusion-criteria
                exclusion-criteria
                recruitment-rate))
  (propose-analysis paper-1)
     ; => '(tables
     ;      (logistic regression))
\end{verbatim}
\end{frame}

\begin{frame}{Example 3: The Round-trip\ldots} 
  \label{example3}
  The first examples describe ``ideas $\rightarrow$ code''

  Consider the last time you read someone else's implementation of a
  statistical procedure (i.e. R package code).  When you read the
  code, could you see:
  \begin{itemize}
  \item the assumptions used?
  \item the algorithm implemented?
  \item practical guidance for when you might select the algorithm
    over others? 
  \item practical guidance for when you might select the
    implementation over others? 
  \end{itemize}
  These are usually components of any reasonable journal article.
  \textit{(Q: have you actually read an R package that wasn't yours?)}
\end{frame}

\begin{frame}{Exercise left to the reader!}

  (aside: I have been looking at the \textbf{stats} and \textbf{lme4}
  packages recently -- \textit{for me}, very clear numerically, much
  less so statistically)
\end{frame}



\subsection{Literate Programming is insufficient}

\begin{frame}{Literate Statistical Practice.}
  \begin{enumerate}
  \item Literate Programming applied to data analysis (Rossini, 1997/2001)
  \item among the \alert{most annoying} techniques to integrate into
    work-flow if one is not perfectly methodological.
  \item Some tools:
    \begin{itemize}
    \item ESS: supports interactive creation of literate programs.
    \item Sweave: tool which exemplifies reporting context; odfWeave
      primarily simplifies reporting.
    \item Roxygen: primarily supports a literate programming
      documentation style, not a literate data analysis programming
      style. 
  \end{itemize}
  \item ROI demonstrated in specialized cases: BioConductor.
  \item \alert{usually done after the fact} (final step of work-flow)
    as a documentation/computational reproducibility technique, rarely
    integrated into work-flow.
  \end{enumerate}
  Many contributors:
  Knuth, Claerbout, Carey, de Leeuw, Leisch, Gentleman, Temple-Lang,
  \ldots{}
\end{frame}

\begin{frame}
  \frametitle{Literate Programming}
  \framesubtitle{Why isn't it enough for Data Analysis?}

  Only 2 contexts: (executable) code and documentation.  Fine for
  application programming,  but for data analysis, we could benefit
  from:
  \begin{itemize}
  \item classification of statistical procedures
  \item descriptions of assumptions
  \item pragmatic recommendations
  \item inheritance of structure through the work-flow of a
    statistical methodology or data analysis project
  \item datasets and metadata
  \end{itemize}
  Concept: ontologies describing mathematical assumptions, applications
  of methods, work-flow, and statistical data structures can enable
  machine communication.
  
  (i.e. informatics framework ala biology)
\end{frame}


\begin{frame}{Communication in Statistical Practice}{\ldots is essential for \ldots}
  \begin{itemize}
  \item finding
  \item explanations
  \item agreement
  \item receiving information
  \end{itemize}
  \alert{``machine-readable'' communication/computation lets the
    computer help} \\
  Semantic Web is about ``machine-enabled computability''.
\end{frame}

\begin{frame}  \frametitle{Semantics}
  \framesubtitle{One definition: description and context}

  Interoperability is the key, with respect to
  \begin{itemize}
  \item ``Finding things''
  \item Applications and activities with related functionality
    \begin{itemize}
    \item moving information from one state to another (paper, journal
      article, computer program)
    \item computer programs which implement solutions to similar tasks
    \end{itemize}
  \end{itemize}
\end{frame}


\begin{frame}{Statistical Practice is somewhat restricted}
  {...but in a good sense, enabling potential for semantics...}

  There is a restrictable set of intended actions for what can be done
  -- the critical goal is to be able to make a difference by
  accelerating activities that should be ``computable'':
  \begin{itemize}
  \item restricted natural language processing
  \item mathematical translation
  \item common description of activities for simpler programming/data
    analysis (S approach to objects and methods)
  \end{itemize}
  R is a good basic start (model formulation approach, simple
  ``programming with data'' paradigm); we should see if we can do
  better!
\end{frame}

\begin{frame}{Computable and Executable Statistics requires}

  \begin{itemize}
  \item approaches to describe data and metadata (``data'')
    \begin{itemize}
    \item semantic WWW
    \item metadata management and integration, driving
    \item data integration
    \end{itemize}
  \item approaches to describe data analysis methods (``models'')
    \begin{itemize}
    \item quantitatively: many ontologies (AMS, etc), few meeting
      statistical needs.
    \item many substantive fields have implementations
      (bioinformatics, etc) but not well focused.
    \end{itemize}
  \item approaches to describe the specific form of interaction
    (``instances of models'')
    \begin{itemize}
    \item Original idea behind ``Literate Statistical Analysis''.
    \item That idea is suboptimal, more structure needed (not
      necessarily built upon existing...).
    \end{itemize}
  \end{itemize}
\end{frame}

\subsection{Common Lisp Statistics}

\begin{frame}
  \frametitle{Interactive Programming}
  \framesubtitle{Everything goes back to being Lisp-like}
  \begin{itemize}
  \item Interactive programming (as originating with Lisp): works
    extremely well for data analysis (Lisp being the original
    ``programming with data'' language).
  \item Theories/methods for how to do this are reflected in styles
    for using R.
  \end{itemize}
\end{frame}
 
\begin{frame}[fragile]
  \frametitle{Lisp}

  Lisp (LISt Processor) is different than most high-level computing
  languages, and is very old (1956).  Lisp is built on lists of things
  which are evaluatable.
\begin{verbatim}
(functionName data1 data2 data3)
\end{verbatim}
  or ``quoted'':
\begin{verbatim}
'(functionName data1 data2 data3)
\end{verbatim}
  which is shorthand for 
\begin{verbatim}
(list functionName data1 data2 data3)
\end{verbatim}
  The difference is important -- lists of data (the second/third) are
  not (yet?!) functions applied to (unencapsulated lists of) data (the first).
\end{frame}

\begin{frame}
  \frametitle{Features}
  \begin{itemize}
  \item Data and Functions semantically the same
  \item Natural interactive use through functional programming with
    side effects
  \item Batch is a simplification of interactive -- not a special mode!
  \end{itemize}
\end{frame}



\begin{frame}[fragile]{Representation: XML and Lisp}{executing your data}
  Many people are familiar with XML: 
\begin{verbatim}
<name phone="+41793674557">Tony Rossini</name>
\end{verbatim}
  which is shorter in Lisp:
\begin{verbatim}
(name "Tony Rossini" :phone "+41613674557")
\end{verbatim}
  \begin{itemize}
  \item Lisp ``parens'', universally hated by unbelievers, are
    wonderful for denoting when a ``concept is complete''.
  \item Why can't your data self-execute?
  \end{itemize}
\end{frame}

\begin{frame}[fragile]{Numerics with Lisp}
  \begin{itemize}
  \item addition of rational numbers and arithmetic
  \item example for mean
\begin{verbatim}
 (defun mean (x)
    (checktype x 'vector-like)
    (/ (loop for i from 0 to (- (nelts *x*) 1)
	  summing (vref *x* i))
       (nelts *x*)))
\end{verbatim}
  \item example for variance
\begin{verbatim}
(defun variance (x)
  (let ((meanx (mean x))
	(nm1 (1- (nelts x))))
     (/ (loop for i from 0 to nm1
	   summing (power (- (vref *x* i) meanx) 2)
        nm1))))
\end{verbatim}
  \item But through macros, \verb+(vref *x* i)+ could be
    \verb+#V(X[i])+ or your favorite syntax.
  \end{itemize}
  
\end{frame}


\begin{frame}{Common Lisp Statistics 1}
  \begin{itemize}
  \item Originally based on LispStat (reusability)
  \item Re-factored structure (some numerics worked with a 1990-era code base). 
  \item Current activities:
    \begin{enumerate}
    \item numerics redone using CFFI-based BLAS/LAPLACK (cl-blapack)
    \item matrix interface based on MatLisp
    \item starting design of a user interface system (interfaces,
      visuals).
    \item general framework for model specification (regression,
      likelihood, ODEs)
    \item general framework for algorithm specification (bootstrap,
      MLE, algorithmic data anaylsis methods).
    \end{enumerate}
  \end{itemize}
\end{frame}

\begin{frame}{Common Lisp Statistics 2}

  \begin{itemize}
  \item Implemented using SBCL.  Contributed fixes for
    Clozure/OpenMCL. Goal to target CLISP
  \item Supports LispStat prototype object system
  \item Package-based design -- only use the components you need, or
    the components whose API you like.
  \end{itemize}
\end{frame}

\section{Discussion}

\begin{frame}
  \frametitle{Outlook}
  \begin{itemize}
  \item Semantics and Computability have captured a great deal of
    attention in the informatics and business computing R\&D worlds
  \item Statistically-driven Decision Making and Knowledge Discovery
    is, with high likelihood, the next challenging stage after data
    integration.
  \item Statistical practice (theory and application) can be enhanced,
    made more efficient, providing  increased benefit to organizations
    and groups using appropriate methods.
  \item Lisp as a language, shares characteristics of both Latin
    (difficult dead language useful for classical training) and German
    (difficult living language useful for general life).  Of course,
    for some people, they are not difficult.
  \end{itemize}

\end{frame}

\begin{frame}
  The research program described in this talk is currently driving the
  design of CommonLisp Stat, which leverages concepts and approaches
  from the dead and moribund LispStat project.

  \begin{itemize}
  \item \url{http://repo.or.cz/w/CommonLispStat.git/}
  \item \url{http://www.github.com/blindglobe/}
  \end{itemize}

\end{frame}
\begin{frame}{Final Comment}

  \begin{itemize}
  \item In the Pharma industry, it is all about getting the right
    drugs to the patient faster.  Data analysis systems seriously
    impact this process, being potentially an impediment or an
    accelerator.

    \begin{itemize}
    \item \alert{Information technologies can increase the efficiency
        of statistical practice}, though innovation change management
      must be taking into account.  (i.e. Statistical practice, while
      considered by some an ``art form'', can benefit from
      industrialization).
    \item \alert{Lisp's features match the basic requirements we need}
      (dichotomy: programs as data, data as programs).  Sales pitch,
      though...
    \item Outlook: Lots of work and experimentation to do!
    \end{itemize}
  \item {\tiny Gratuitous Advert: We are hiring, have student
      internships (undergrad, grad students), and a visiting faculty
      program.  Talk with me if possibly interested.}
  \end{itemize}
\end{frame}


% % All of the following is optional and typically not needed. 
% \appendix


% \section<presentation>*{\appendixname}


% \begin{frame} \frametitle{Complements and Backup}
%   No more, stop here.  Questions?  (now or later).
% \end{frame}

% \begin{frame}{The Industrial Challenge.}{Getting the Consulting Right.}
%   % - A title should summarize the slide in an understandable fashion
%   %   for anyone how does not follow everything on the slide itself.

%   \begin{itemize}
%   \item Recording assumptions for the next data analyst, reviewer.
%     Use \texttt{itemize} a lot.
%   \item
%     Use very short sentences or short phrases.
%   \end{itemize}
% \end{frame}


% \begin{frame}{The Industrial Challenge.}{Getting the Right Research Fast.}
%   % - A title should summarize the slide in an understandable fashion
%   %   for anyone how does not follow everything on the slide itself.

%   \begin{itemize}
%   \item
%     Use \texttt{itemize} a lot.
%   \item
%     Use very short sentences or short phrases.
%   \end{itemize}
% \end{frame}


% \begin{frame}{Explicating the Work-flow}{QA/QC-based improvements.}


% \end{frame}

% \section{Motivation}

% \subsection{IT Can Speed up Deliverables in Statistical Practice}

% \begin{frame}{Our Generic Work-flow and Life-cycle}
%   {describing most data analytic activities}
%   Workflow:
%   \begin{enumerate}
%   \item Scope out the problem
%   \item Sketch out a potential solution
%   \item Implement until road-blocks appear
%   \item Deliver results 
%   \end{enumerate}

%   Lifecycle:
%   \begin{enumerate}
%   \item paper sketch
%   \item 1st e-draft of text/code/date (iterate to \#1, discarding)
%   \item cycle through work
%   \item publish
%   \item ``throw-away''
%   \end{enumerate}
%   but there is valuble information that could enable the next
%   generation!
% \end{frame}

% \begin{frame}[fragile]{Paper $\rightarrow$ Computer  $\rightarrow$ Article $\rightarrow$ Computer}{Cut and Paste makes for large errors.}
%   \begin{itemize}
%   \item Problems in a regulatory setting
%   \item Regulatory issues are just ``best practices''
%   \end{itemize}

%   Why do we ``copy/paste'', or analogously, restart our work?

%   pro:
%   \begin{itemize}
%   \item every time we repeat, we reinforce the idea in our brain
%   \item review of ideas can help improve them
%   \end{itemize}
%   con:
%   \begin{itemize}
%   \item inefficiency
%   \item introduction of mistakes
%   \item loss of historical context
%   \item changes to earlier work (on a different development branch)
%     can not propagate. 
%   \end{itemize}
% \end{frame}

% \section{Semantics and Statistical Practice}


% \begin{frame}
%   \frametitle{Statistical Activity Leads to Reports}
%   \framesubtitle{You read what you know, do you understand it?}

%   How can we improve the communication of the ideas we have?

%   Precision of communication?

% \end{frame}



% \begin{frame}  \frametitle{Communication Requires Context}
%   \framesubtitle{Intentions imply more than one might like...}

%   \begin{itemize}
%   \item Consideration of what we might do
%   \item Applications with related functionality
%   \end{itemize}
% \end{frame}



% \begin{frame}
%   \frametitle{Design Patterns}
%   \framesubtitle{Supporting Work-flow Transitions}
  
%   (joint work with H Wickham): The point of this research program is
%   not to describe what to do at any particular stage of work, but to
%   encourage researchers and practitioners to consider how the
%   translation and transfer of information between stages so that work
%   is not lost.

%   Examples of stages in a work-flow:
%   \begin{itemize}
%   \item planning, execution, reporting;
%   \item scoping, illustrative examples or counter examples, algorithmic construction,
%     article writing.
%   \item descriptive statistics, preliminary inferential analysis,
%     model/assumption checking, final inferential analysis,
%     communication of scientific results
%   \end{itemize}
%   Description of work-flows is essential to initiating discussions on
%   quality/efficiency of approaches to work.
% \end{frame}

% \section{Design Challenges}

% \begin{frame}
%   \frametitle{Activities are enhanced by support}

%   \begin{itemize}
%   \item Mathematical manipulation can be enhanced by symbolic
%     computation
%   \item Statistical programming can be enabled by examples and related
%     algorithm implementation
%   \item Datasets, to a limited extent, can self-describe.
%   \end{itemize}
% \end{frame}

% \begin{frame}
%   \frametitle{Executable and Computable Science}
  
%   Use of algorithms and construction to describe how things work.

%   Support for agent-based approaches
% \end{frame}


% \begin{frame}
%   \frametitle{What is Data?  Metadata?}

%   Data: what we've observed

%   MetaData: context for observations, enables semantics.
% \end{frame}




% % \begin{frame}[fragile]
% %   \frametitle{Defining Variables}
% %   \framesubtitle{Setting variables}
% % \begin{verbatim}
% % (setq <variable> <value>)
% % \end{verbatim}
% %   Example:
% % \begin{verbatim}
% % (setq ess-source-directory
% %       "/home/rossini/R-src")
% % \end{verbatim}
% % \end{frame}

% % \begin{frame}[fragile]
% %   \frametitle{Defining on the fly}
% % \begin{verbatim}
% % (setq ess-source-directory
% %    (lambda () (file-name-as-directory
% %          (expand-file-name
% %            (concat (default-directory)
% %                    ess-suffix "-src")))))
% % \end{verbatim}
% %   (Lambda-expressions are anonymous functions, i.e. ``instant-functions'')
% % \end{frame}


% % \begin{frame}[fragile]
% %   \frametitle{Function Reuse}
% %   By naming the function, we could make the previous example reusable
% %   (if possible):
% % \begin{verbatim}
% % (defun my-src-directory ()
% %       (file-name-as-directory
% %          (expand-file-name
% %            (concat (default-directory)
% %                    ess-suffix "-src"))))
% % \end{verbatim}
% %   Example:
% % \begin{verbatim}
% % (setq ess-source-directory (my-src-directory))
% % \end{verbatim}
% % \end{frame}


% % \begin{frame}
% %   \frametitle{Equality Among Packages}
% %   \begin{itemize}
% %   \item more/less equal can be described specifically through
% %     overriding imports.
% %   \end{itemize}
% % \end{frame}


% \subsection<presentation>*{For Further Reading}

% \begin{frame}[allowframebreaks]
%   \frametitle<presentation>{Related Material}
    
%   \begin{thebibliography}{10}
    
%   \beamertemplatebookbibitems
%   % Start with overview books.

%   \bibitem{LispStat1990}
%     L.~Tierney
%     \newblock {\em LispStat}.
 
%   \beamertemplatearticlebibitems
%   % Followed by interesting articles. Keep the list short. 

%   \bibitem{Rossini2001}
%     AJ.~Rossini
%     \newblock Literate Statistical Practice
%     \newblock {\em Proceedings of the Conference on Distributed
%       Statistical Computing}, 2001.

%   \bibitem{RossiniLeisch2003}
%     AJ.~Rossini and F.~Leisch
%     \newblock Literate Statistical Practice
%     \newblock {\em Technical Report Series, University of Washington
%       Department of Biostatistics}, 2003.

%   \beamertemplatearrowbibitems
%   % Followed by interesting articles. Keep the list short. 

%   \bibitem{CLS}
%     Common Lisp Stat, 2008.
%     \newblock \url{http://repo.or.cz/CommonLispStat.git/}

%   \end{thebibliography}
% \end{frame}

\end{document}
