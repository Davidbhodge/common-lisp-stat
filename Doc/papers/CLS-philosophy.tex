\documentclass{article}

\title{CLS: an approach for a new statistical system}
\author{AJ Rossini}
\date{\today}

\begin{document}

\maketitle

\section{Introduction}
\label{sec:intro}

Statisticians who use a computer for data analysis invariably take one
of two approaches (considered in the extremes here for illustration): 
\begin{enumerate}
\item the \emph{FORTRAN} approach of coding numerical and algorithmic
  information into the computer program code used for the data
  analysis, or 
\item the \emph{GUI} approach, via Microsoft Excel, SPSS, Minitab, and
  similar approaches, where tasks are facilitated, sometimes with
  accompanying workflow support.
\end{enumerate}
Both approaches have co-existed since the early 80s, with the FORTRAN
approach dating back to the dawn of the computing era.

\section{Components of a procedure}
\label{sec:components}

define a statistical procedure as a decision-making approach which
entails the intertwining of formal and informal structure.   

Components:
\begin{enumerate}
\item \label{statproc-decision} Decision to make
\item \label{statproc-assessment} Assessment approach to use
  (some are inherently different, others just look different)
\item \label{statproc-normalization} Normalization of the problem for
  assessment/comparison with other reference behaviours
\item \label{conclusion} Type of conclusion desired, and instance of
  that conclusion (when data is present)
\end{enumerate}

This forms an \textit{abstract class} of a procedure, which can be
represented by a real class, which can then be instantiated through
the application of data.

\subsection{Decision}
\label{sec:components:decision}

By example, consider the t-test as an instance of a procedure,
representing the general class of testing hypotheses surrounding 2
means.  Related would be formal likelihood tests with distributions,
the superspace/classes from regression and ANOVA.
Questions could be:
\begin{itemize}
\item are the 2 means the same?
\item what is the difference?
\item what is the strength of the difference?
\end{itemize}

\subsection{Core Assessment}
\label{sec:components:assessment}

This is the construction of the model and parameters that would be
used to form the term used to make the assessment.  Here, we could
consider 
\begin{equation}
  \label{eq:assess:ex:1}
  \hat{E}[Y|G=1] - \hat{E}[Y|G=0]
\end{equation}
as the fundamental quantity to compare.    This can arise from many
sources such as regression models
\begin{equation}
  \label{eq:assess:ex:2}
  Y = \mu + \beta G + \epsilon \\
  E[\epsilon] = 0
\end{equation}
or 
\begin{equation}
  \label{eq:assess:ex:2}
  E[Y|G] = \mu + \beta G 
\end{equation}

\subsection{Normalized Behavior}
\label{sec:components:normbeh}
Let $X=(Y,G)$ from above, the whole data.

empirical adjustment:
\begin{equation}
  \label{eq:norm:ex:1}
  \frac{ \hat\mu_1 - \hat\mu_0}%
  {\hat{SE}(\hat\mu_1 - \hat\mu_0)}
\end{equation}
or regression-model-based:
\begin{equation}
  \label{eq:norm:ex:2}
  \frac{ \hat\beta}%
  {\hat{SE}(\hat\beta)}
\end{equation}
or likelihood-model-based: (FIXME!)
\begin{equation}
  \label{eq:norm:ex:3}
  -2 \log \frac{ L(\hat\beta|X)}%
  {L(0|X)}
\end{equation}
or score-model-based:
\begin{equation}
  \label{eq:norm:ex:4}
  \cal{I}^{-1}(\beta=0,X) S(\beta=0,X) 
\end{equation}

\subsection{Conclusion Desired}
\label{sec:component:conclusion}

Value or Range on the Target Scale (existing parameter describing
data-oriented substantive model)

Translation of Value/Range on the Decision Scale (what to do, what to
decide about the problem, i.e. in a testing framework).

\section{Class Implementation}
\label{sec:class}


\section{Discussion}
\label{sec:disc}



\end{document}
